\documentclass[uplatex,dvipdfmx]{jsarticle}
\usepackage{listings,jvlisting}
\usepackage{color}
\usepackage{xr}

\lstset{
  backgroundcolor=\color[rgb]{0.9,0.9,0.9},
  frame=single,
  basicstyle=\ttfamily
}
\makeatletter
\newcommand*{\addFileDependency}[1]{% argument=file name and extension
  \typeout{(#1)}
  \@addtofilelist{#1}
  \IfFileExists{#1}{}{\typeout{No file #1.}}
}
\makeatother

\newcommand*{\myexternaldocument}[1]{%
    \externaldocument{#1}%
    \addFileDependency{#1.tex}%
    \addFileDependency{#1.aux}%
}
\myexternaldocument{../categoryTheoryIntro/CategoryTheoryIntro}

\newcommand{\pr}[1]{\colorbox[rgb]{0.9,0.9,0.9}{\lstinline{#1}}}
\newcommand{\functype}[2]{\pr{#1 -> #2}}
\newcommand{\refcti}[1]{\cite{cti}\ref{#1}}
\newcommand{\fpmor}[3]{\pr{#1 :: #2 -> #3}}

\input{../categoryTheoryIntro/preemble.tex}
\newtheorem{lemma}[proof]{補題}

\begin{document}
  \title{丹原Double圏同値の証明}
  \maketitle
  モノイダル積の記号$\otimes$は省略する。
  \begin{define}[Opticsの圏]\label{def-cat-of-optics}
    対称モノイダル圏$\cat{C}$上の圏$\cat{Optic}$を以下のように定義する。
    \begin{quote}
			\begin{mydescription}
				\item[対象] 対象の集合を$\obj{Optic}=\obj{C}\times \obj{C}$とする。
				\item[射] 任意の対象$(A,A'), (S,S')$に対する射集合を\[\arset{Optic}{(A,A')}{(S,S')}=\int^{M}\arset{C}{S}{MA}\times\arset{C}{MA'}{S'}\]とする。また\[\arset{Optic}{(A,A')}{(S,S')}\cong\int^M\arset{C^{op}\times C}{(MA,MA')}{(S,S')}\]も成り立つ。
				\item[射の合成] {\[\mor{\circ}{\int^{N} \arset{C}{T}{NS}\times\arset{C}{NS'}{T'}\times \int^{M}\arset{C}{S}{MA}\times \arset{C}{MA'}{S'}}{\int^{MN}\arset{C}{T}{(MN)A}\times \arset{C}{(MN)A'}{T'}}\]}を射の合成を行う写像とみなすということである。またこの写像は、ある対象$M,N$に対して
        \[\tuple{l',r'}\circ_{MN}\tuple{l,r}=\tuple{a_{NMA}^{-1}\circ(Nl)\circ l',r'\circ(Nr)\circ a_{NMA'}^{-1}}\]
        となるような操作である。
				\item[恒等射の存在] 任意の対象$(S,S')$に対して\[\tuple{\lambda_S^{-1},\lambda_{S'}}:\int^{M}\arset{C}{S}{MS}\times \arset{C}{MS'}{S'}\]を恒等射とする。
				\item[結合律]任意のOptic$\mor{\tuple{l'',r''}}{(T,T')}{(R,R')},\ \mor{\tuple{l',r'}}{(S,S')}{(T,T')},\ \mor{\tuple{l,r}}{(A,A')}{(S,S')}$に対して、
        \[(\tuple{l'',r''}\circ\tuple{l',r'})\circ\tuple{l',r'}=\tuple{l'',r''}\circ(\tuple{l',r'}\circ\tuple{l,r})\]が成り立つ。これは射の合成の定義により展開していけば簡単に示せる。
				\item[単位元律]任意のOptic$\mor{\tuple{l,r}}{(A,A')}{(S,S')}$に対して、
        \[\tuple{l,r}\circ\tuple{\lambda_A^{-1},\lambda_{A'}}=\tuple{l,r}\]
        \[\tuple{\lambda_S^{-1},\lambda_{S'}}\circ\tuple{l,r}=\tuple{l,r}\]
        が成り立つ。これはコエンドの超自然性とモノイダル圏の三角恒等式から示せる。
			\end{mydescription}
		\end{quote}
  \end{define}
  \begin{define}[丹原加群の圏]
    対称モノイダル圏$\cat{C}$に対する圏$\cat{Tamb}$を以下のように定義する。
    \begin{quote}
			\begin{mydescription}
				\item[対象]
				以下の図式を可換にするようなProfunctor\ $\profunctor{P}{C}{C}$と自然変換$\zeta$の組$(P,\zeta)$を対象とする。
        \begin{center}
          \begin{tikzpicture}[auto]
            \node (P) at (0, 0) {$P(A,A')$};
            \node (TP) at (6, 0) {$P(MA,MA')$};
            \node (TP2) at (0, -2) {$P((NM)A,(NM)A')$};
            \node (TTP) at (6, -2) {$P(N(MA),N(MA'))$};
            \draw[->] (P) to node{$\zeta_{M,A,A'}$}(TP);
            \draw[->] (P) to node{$\zeta_{NM,A,A'}$}(TP2);
            \draw[->] (TP) to node{$\Theta \zeta_N$}(TTP);
            \draw[->] (TP2) to node{$P(a^{-1},a)$}(TTP);
          \end{tikzpicture}
        \end{center}
        \begin{center}
          \begin{tikzpicture}[auto]
            \node (P) at (0, 0) {$P(A,A')$};
            \node (TP) at (4, 0) {$P(IA,IA')$};
            \node (P2) at (4, -2) {$P(A,A')$};
            \draw[->] (P) to node{$\zeta_{I,A,A'}$}(TP);
            \draw[->] (TP) to node{$P(\lambda^{-1}_A,\lambda_{A'})$}(P2);
            \draw[->] (P) to(P2);
          \end{tikzpicture}
        \end{center}
				\item[射]対象$(P,\zeta)$、$(Q,\eta)$の間の射を、
        以下の図式を可換にするような自然変換$\nat{\alpha}{P}{Q}$とする。
  
        \begin{center}
          \begin{tikzpicture}[auto]
            \node (P) at (0, 0) {$P(A,A')$};
            \node (Q) at (0, -2) {$Q(A,A')$};
            \node (TP) at (3, 0) {$\int^{M}P(MA, MA)$};
            \node (TQ) at (3, -2) {$\int^{M}Q(MA, MA')$};
            \node (catpr) at (1.5, 1) {$\cat{Set}$};
  
            \draw[->] (P) to node{$\alpha_{A,A'}$}(Q);
            \draw[->] (P) to node{$\zeta_{M,A,A'}$}(TP);
            \draw[->] (TP) to node{$\int\alpha_{A,A'}$}(TQ);
            \draw[->] (Q) to node{$\mu_{M,A,A'}$}(TQ);
  
            \node (P) at (6, 0) {$(P,\zeta)$};
            \node (Q) at (6, -2) {$(Q,\mu)$};
            \draw[->] (P) to node{$\alpha$}(Q);
            \node (cattam) at (6, 1) {$\cat{Tamb}$};
  
          \end{tikzpicture}
        \end{center}
          
				\item[射の合成]射$\alpha,\ \int\alpha$の自然変換の垂直合成をそのまま$\cat{Tamb}$における射の合成とする。
				\item[恒等射の存在]
				射の合成と同様に恒等自然変換を$\cat{Tamb}$の恒等射とする。
				\item[結合律] 省略
				\item[単位元律] 省略
			\end{mydescription}
		\end{quote}
  \end{define}
  \begin{prop}[丹原Double圏同値]
    任意の対称モノイダル圏$\cat{C}$上において$\cat{Tamb}\simeq\incat{Optic}{Set}$
  \end{prop}
  \begin{lemma}あるProfunctor $\profunctor{P}{C}{C}$に対して
    \[\int_{AA'M}\inset{P(A,A')}{P(MA,MA')}\cong\int_{AA'SS'M}\inset{\arset{C^{op}\times C}{MA, MA'}{S,S'}}{\inset{P(A,A')}{P(S,S')}}\]
  \end{lemma}
  \begin{proof}
    エンドを取る操作は関手であるから、
    \[\inset{P(B,B')}{P(MA,NA')}\cong\int_{SS'}\inset{\arset{C^{op}\times C}{MA, NA'}{S,S'}}{\inset{P(B,B')}{P(S,S')}}\]が$M,A,A',B,B'$に対して自然に成り立つことを示せばよい。\\
    同型性については$F(X,X'):=\inset{P(A,A')}{P(MX,NX')}$とすると米田の補題\[FA\cong\int_M \inset{\arset{C}{A}{X}}{FX}\]より成り立つ。また$M,N,A,A'$に対する自然性は米田の補題の$F$に対する自然性から、$B,B'$に対する自然性は米田の補題の$A$に対する自然性から成り立つ。
  \end{proof}
\end{document}