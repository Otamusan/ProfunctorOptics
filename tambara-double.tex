\documentclass[uplatex,dvipdfmx]{jsarticle}
\usepackage{listings,jvlisting}
\usepackage{color}
\usepackage{xr}

\lstset{
  backgroundcolor=\color[rgb]{0.9,0.9,0.9},
  frame=single,
  basicstyle=\ttfamily
}
\makeatletter
\newcommand*{\addFileDependency}[1]{% argument=file name and extension
  \typeout{(#1)}
  \@addtofilelist{#1}
  \IfFileExists{#1}{}{\typeout{No file #1.}}
}
\makeatother

\newcommand*{\myexternaldocument}[1]{%
    \externaldocument{#1}%
    \addFileDependency{#1.tex}%
    \addFileDependency{#1.aux}%
}
\myexternaldocument{../categoryTheoryIntro/CategoryTheoryIntro}

\newcommand{\pr}[1]{\colorbox[rgb]{0.9,0.9,0.9}{\lstinline{#1}}}
\newcommand{\functype}[2]{\pr{#1 -> #2}}
\newcommand{\refcti}[1]{\cite{cti}\ref{#1}}
\newcommand{\fpmor}[3]{\pr{#1 :: #2 -> #3}}

\input{../categoryTheoryIntro/preemble.tex}
\newtheorem{lemma}[proof]{補題}
\newcommand{\kintou}[2]{\leavevmode\hbox to #1{%
    \kanjiskip\z@ plus 1fill minus 1fill\xkanjiskip\kanjiskip #2\hfil}}
\begin{document}
  \title{丹原Double圏同値の証明}
  \maketitle
  \pagebreak
  \begin{prop}[丹原Double圏同値]
    任意のモノイダル圏$\cat{C}$上において$\cat{Tamb}\simeq\incat{Optic}{Set}$
  \end{prop}
  \begin{lemma}あるProfunctor $\profunctor{P}{C}{C}$に対して
    \[\int_{AA'M}\inset{P(A,A')}{P(M\otimes A,M\otimes A')}\cong\int_{AA'SS'M}\inset{\arset{C^{op}\times C}{M\otimes A, M\otimes A'}{S,S'}}{\inset{P(A,A')}{P(S,S')}}\]
  \end{lemma}
  \begin{proof}
    エンドを取る操作は関手であるから、
    \[\inset{P(B,B')}{P(M\otimes A,N\otimes A')}\cong\int_{SS'}\inset{\arset{C^{op}\times C}{M\otimes A, N\otimes A'}{S,S'}}{\inset{P(B,B')}{P(S,S')}}\]が$M,A,A',B,B'$に対して自然に成り立つことを示せばよい。\\
    同型性については$F(X,X'):=\inset{P(A,A')}{P(M\otimes X,N\otimes X')}$とすると米田の補題\[FA\cong\int_M \inset{\arset{C}{A}{X}}{FX}\]より成り立つ。また$M,N,A,A'$に対する自然性は米田の補題の$F$に対する自然性から、$B,B'$に対する自然性は米田の補題の$A$に対する自然性から成り立つ。
  \end{proof}
  またこの時の同型射を\[\mor{\phi}{\int_{AA'M}\inset{P(A,A')}{P(M\otimes A,M\otimes A')}}{\int_{AA'SS'M}\inset{\arset{C^{op}\times C}{M\otimes A, M\otimes A'}{S,S'}}{\inset{P(A,A')}{P(S,S')}}}\]\[\mor{\phi^{-1}}{\int_{AA'SS'M}\inset{\arset{C^{op}\times C}{M\otimes A, M\otimes A'}{S,S'}}{\inset{P(A,A')}{P(S,S')}}}{\int_{AA'M}\inset{P(A,A')}{P(M\otimes A,M\otimes A')}}\]とし、それぞれ\[\phi(\zeta) = \lambda\tuple{l,r}.P(l,r)\circ \zeta\]
  \[\phi^{-1}(F) = F(id,id)\]となる。
  \begin{lemma}\label{lemma-tamb-is-functor}
    ある丹原加群$(P,\zeta)$に対して、$\phi(\zeta)$はある関手$\functor{F}{Optic}{Set}$の射関数である。
  \end{lemma}
  \begin{proof}
    以降の証明では一部の添え字とモノイダル積の記号を省略する。
    仮定より\[\zeta_N\circ\zeta_M = P(a^{-1},a)\circ\zeta_{N\otimes M}\]\[P(\lambda^{-1},\lambda)\circ\zeta_I = id\]であり、それによって以下の図式が可換になることを示せばよい。
    \begin{center}

      \begin{tikzpicture}[auto]

        \node (ct) at (0, 0) {\scriptsize
        $\arset{(C^{op}\times C)}{NS,NS'}{T,T'}\times\arset{(C^{op}\times C)}{MA,MA'}{S,S'}$};

        \node (pt) at (8, 0) {\scriptsize$\inset{P(S,S')}{P(T,T')}\times\inset{P(A,A')}{P(S,S')}$};

        \node (c) at (0, -2) {\scriptsize
        $\arset{(C^{op}\times C)}{(NM)A,(NM)A'}{T,T'}$};
        \node (p) at (8, -2) {\scriptsize$\inset{P(A,A')}{P(T,T')}$};
        \draw[->] (ct) to node{$\phi(\zeta_N)\times \phi(\zeta_M)$}(pt);
        \draw[->] (c) to node[swap]{$\phi(\zeta_{NM})$}(p);
        \draw[->] (ct) to node[swap]{$\circ_{\cat{Optic}}$}(c);
        \draw[->] (pt) to node[swap]{$\circ$}(p);

      \end{tikzpicture}
    \end{center}
    \begin{center}

      \begin{tikzpicture}[auto]

        \node (ca) at (0, 0) {$\arset{(C^{op}\times C)}{MA,MA'}{A,A'}$};
        \node (a) at (5, 0) {$\inset{P(A,A')}{P(A,A')}$};
        \node (1) at (5, -2) {$1$};
        \draw[->] (ca) to node{$\phi(\zeta_M)$}(a);
        \draw[->] (1) to node[swap]{$j_{P(A,A)}$}(a);
        \draw[->] (1) to node{$\tuple{\lambda^{-1},\lambda}$}(ca);

      \end{tikzpicture}
    \end{center}

    前者の図式の左上に任意のOpticの組 $\tuple{l',r',l,r}$を適用すると、それぞれ\[P(a^{-1}\circ Nl\circ l',r'\circ Nr\circ a)\circ\zeta_{NM}\]\[P(l',r')\circ\zeta_N\circ P(l,r)\circ \zeta_M\]となるから、これらが等しいことを示す。
    \begin{align*}
      P(l',r')\circ\zeta_N\circ P(l,r)\circ \zeta_M &=P(l',r')\circ P(Nl,Nr)\circ\zeta_N\circ \zeta_M&\text{($\zeta$の自然性)}\\
      &=P(l',r')\circ P(Nl,Nr)\circ P(a^{-1},a) \circ\zeta_{NM}&\text{(仮定)}\\
      &=P(a^{-1}\circ Nl\circ l',\ r'\circ Nr\circ a)\circ\zeta_{NM}\\
    \end{align*}
    同様に後者の図式から$P(\lambda^{-1},\lambda)\circ\zeta_M$を示せばよいが、これはそのまま$1$の元を適用すれば示せる。
  \end{proof}
  \begin{lemma}
    あるProfunctor$\profunctor{P}{C}{C}$に対して$F(A,B)=P(A,B)$となるような任意の関手$\functor{F}{Optic}{Set}$が存在するとする。
    すると$F$の射関数$F$に対して、$(P,\phi^{-1}(F))$が丹原加群となる。
  \end{lemma}
  \begin{proof}
    仮定の$F$の関手性から、任意のOptic$\tuple{l,r}:\mathrm{Optic}(A,A',S,S'),\ \tuple{l',r'}:\mathrm{Optic}(S,S',T,T')$に対して、\[F_N(l',r')\circ F_M(l,r) = F_{NM}(a^{-1}\circ Nl\circ l',\ r'\circ Nr\circ a)\]
    \[F_I(\lambda^{-1},\lambda)=id_{P(A,A)}\]であり、それによって以下の図式が可換になることを示せばよい。
      \begin{center}
        \begin{tikzpicture}[auto]
          \node (P) at (0, 0) {$P(A,A')$};
          \node (TP) at (6, 0) {$P(MA,MA')$};
          \node (TP2) at (0, -2) {$P((NM)A,(NM)A')$};
          \node (TTP) at (6, -2) {$P(N(NA),N(MA'))$};
          \draw[->] (P) to node{$\phi(F_M)$}(TP);
          \draw[->] (P) to node{$\phi(F_{NM})$}(TP2);
          \draw[->] (TP) to node{$\phi(F_N)$}(TTP);
          \draw[->] (TP2) to node{$P(a^{-1},a)$}(TTP);
        \end{tikzpicture}
      \end{center}
      \begin{center}
        \begin{tikzpicture}[auto]
          \node (P) at (0, 0) {$P(A,A')$};
          \node (TP) at (4, 0) {$P(IA,IA')$};
          \node (P2) at (4, -2) {$P(A,A')$};
          \draw[->] (P) to node{$\phi(F_I)$}(TP);
          \draw[->] (TP) to node{$P(\lambda^{-1},\lambda)$}(P2);
          \draw[->] (P) to(P2);
        \end{tikzpicture}
      \end{center}
      すなわち、\[F_N(id,id)\circ F_M(id,id) = P(a,a^{-1})\circ F_{NM}(id,id)\]
      \[P(\lambda^{-1},\lambda)\circ F_I(id,id)\]を示せばよい。

      前者については、$l',r',l,r$を恒等射とすると、
      \begin{align*}
        F_N(id,id)\circ F_M(id,id) &=F_{NM}(a^{-1}\circ (N\otimes id)\circ id,\ id\circ (N\otimes id)\circ a) &\text{(仮定)}\\
        &=F(a,a^{-1})
      \end{align*}
  \end{proof}

\end{document}